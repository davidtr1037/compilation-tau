%%%%%%%%%%%%%%%%%%
% DOCUMENT CLASS %
%%%%%%%%%%%%%%%%%%
\documentclass{article}

%%%%%%%%%%%%
% PACKAGES %
%%%%%%%%%%%%
\usepackage{hyperref}
\usepackage{amsmath}
\usepackage{multirow}
\usepackage{placeins}
\usepackage{xcolor}
\usepackage{listings}
\usepackage{xspace}
\usepackage[capitalise]{cleveref}

%%%%%%%%%%%%%%%%%%
% BEGIN DOCUMENT %
%%%%%%%%%%%%%%%%%%

\newcommand{\nr}[1]{{\color{red}[#1]}}
\begin{document}

%%%%%%%%%
% TITLE %
%%%%%%%%%
\title{Exercise 4 (Dataflow Analysis)}

%%%%%%%%%%%%%%%%%%%%%%%%%%%%%
% AUTHOR = COURSE NAME HERE %
%%%%%%%%%%%%%%%%%%%%%%%%%%%%%
%\author{Compilation 0368:3133}

%%%%%%%%%%%%%%%%%%%%%%%%%%%%%%%
% DATE = SUBMISSION DATE HERE %
%%%%%%%%%%%%%%%%%%%%%%%%%%%%%%%
\date{Due 19/4/2024 before 23:59}

%%%%%%%%%
% TITLE %
%%%%%%%%%
\maketitle

\newcommand{\plname}{\textbf{L}\xspace}

%%%%%%%%%%%%%%%%%%%%%%%%%%%
% SECTION :: Introduction %
%%%%%%%%%%%%%%%%%%%%%%%%%%%
\section{Introduction}
We continue our journey of building a compiler
for the invented object-oriented language \plname.
This time, we dive into the realm of the machine-independent optimizer. 
 
\section{Programming Assignment}
The fourth exercise implements a dataflow analysis for \plname programs.
Specifically, you need to implement an uninitialized variable analysis for (a subset of) \plname.
The exercise is roughly divided into three parts as follows:
\begin{enumerate}
\item Recursively traverse the AST to create
an intermediate representation (IR) of the program.
\item Construct a control-flow graph.
\item Compute the results of the analysis using chaotic iterations.
\end{enumerate}

The input for this last exercise is a (single) text file containing an \plname
program in the subset defined below,
and the output is a (single) text file that contains an indication of whether the program does not access uninitialized variables or a list of such variables if it does.

\section{The Input Programs}
\subsection{Syntax}
To simplify the exercise, you may assume that the program consists only of a main function with signature \texttt{void main()}.
Furthermore, \texttt{main} may only use (global and local) variables of type \texttt{int} and the library function \texttt{PrintInt}.

\subsection{Semantics of Initializations}
Recall that both global and local variables may be initialized when defined.
The initial value can be any numerical expression over previously defined variables.
Thus, when these initializations occur, they should be in the same order of appearance as in the original program.
Specifically, before entering main, all global variables with initialized values should be evaluated.
 
\section{Uninitialized Variables Analysis}

One of the most common programming errors is the use of a
variable before its definition. This undefined value may produce incorrect
results, memory violations, unpredictable behaviors, and program failure.
More specifically, \emph{used-before-set} refers to the error occurring when a program uses a variable that has not been assigned an (initialized) value. This uninitialized variable, once used in
a calculation, can be quickly propagated throughout the entire program, and
anything may happen. The program may produce different results each time it
runs, may crash for no apparent reason, or may behave unpredictably~\cite{uninitvar}. 

In this exercise, you are asked to implement a \emph{conservative} compile-time dataflow analysis that detects used-before-set errors.
In class, we have seen a way to implement a used-before-def analysis based on reaching definitions (see, e.g., \cite[Sec. 9.2.4]{dragonbook}).
Here, you are asked to design a slightly more sophisticated analysis where a definition sets a variable \texttt{x} to an initialized value only if the value assigned to \texttt{x}  is initialized.

More formally, a program may have a used-before-set-a-valid-value error\footnote{For brevity, from now on we will refer to such an error as a used-before-set error.} involving variable \texttt{x} if there is a path $p$ from the start of the program to a program point in which an expression using \texttt{x} is evaluated, and path $p$ does not contain an assignment which sets \texttt{x} to an \emph{initialized value}. An assignment $l: \texttt{x} := exp$ in path $p$ sets \texttt{x} to an initialized value if $exp$ consists of constants and variables that were assigned an initialized value prior to $l: \texttt{x} := exp$ in $p$.
Stated differently, assigning a variable with an expression involving an uninitialized value does not initialize that variable.
Your analysis should detect and report all variables possibly involved in used-before-set errors. The following examples clarify these notions. 

The program in \cref{fig:example-immediate} has use-before-set errors involving variables
\texttt{g} and \texttt{y}.
Note that \text{g} is used in an assignment command while \texttt{y} is used as part of a boolean condition.


\begin{figure}
\centering
\begin{lstlisting}[language=C]
int g;
void main() {
  int x;
  int y;
  int z;
  
  x := 1;
  if (y>0)  { z := x + g + 1;  x := x + 1;}
}

// expected output:
// g
// y
\end{lstlisting}
\caption{A program accessing uninitialized local and global variables.}
\label{fig:example-immediate}
\end{figure} 

The program in \cref{fig:example-flow} has use-before-set errors involving variables
\texttt{g}, \texttt{x}, and \texttt{y}.
Note that \texttt{x} is assigned a value before it is used. 
This assignment, however, is to an expression containing \texttt{g}, which is uninitialized at that point. 


\begin{figure}
\centering
\begin{lstlisting}[language=C]
int g;
void main() {
  int x := g * 2;
  int y;
  int z;
  
  g := 1;
  if (y>0)  { z := x + 1; }
}

// expected output:
// g
// x
// y
\end{lstlisting}
\caption{A program accessing uninitialized local and global variables.}
\label{fig:example-flow}
\end{figure} 
 

The program in \cref{fig:example-loop} also has use-before-set errors involving variables
\texttt{g}, \texttt{x}, and \texttt{y}.
Note that the use of \texttt{g} in \texttt{PrintInt(g)} is considered a use-before-set error because the loop body may never execute.
Also, note that although \texttt{y} is involved in two potential use-before-def errors, it is printed only once.

The analysis \emph{should not} try to infer the infeasibility of passes at compile time. For example, we replace the declaration of \texttt{y} in \Cref{fig:example-loop} with 
\texttt{int y := 10;}
then the analysis should still report that variables \texttt{x} and \texttt{g} may be used-before-set, although the body of the loop in the program in \cref{fig:example-loop} is always executed and thus assigns an initialized value to \texttt{g}.


\begin{figure}
\centering
\begin{lstlisting}[language=C]
int g;
void main() {
  int x;
  int y;
  int z;
  
  while (y>0)  { z := x + 1;  g:=1; y:= y-1;}
  PrintInt(g);
}

// expected output:
// g
// x
// y
\end{lstlisting}
\caption{A program accessing uninitialized local and global variables.}
\label{fig:example-loop}
\end{figure} 

The program in \cref{fig:example-ok}  has no use-before-set errors. 


\begin{figure}
\centering
\begin{lstlisting}[language=C]
int g0 := 0;
int g:= g0 + 1;
void main() {
  int x := g * 2 + 3;
  int y := 10;
  int z;
  int w;
  
  while (y>0)  { z := x + 1 + y;  g:=1; y:= y-1;}
  PrintInt(g);
}

// expected output:
// - - OK - -
\end{lstlisting}
\caption{A program which does not access uninitialized variables.}
\label{fig:example-ok}
\end{figure} 

\section{Submission Guidelines}
 
\subsection{Input}
The input for this exercise is a single text file, the input program in the assumed subset of  \plname.

\subsection{Output}
The output is a single text file that contains either the string \texttt{--OK--}
if there are no read accesses from uninitialized variables or a sorted list of the names of the uninitialized variables that were accessed. The exercise's skeleton contains helper functions that format the output.

\subsection{Skeleton}
The skeleton for this exercise can be found \href{https://github.com/davidtr1037/compilation-tau/tree/master/src/ex4-dfa}{here}.
The skeleton contains a partial implementation of the IR, which you may adapt to your needs.

The skeleton also contains code for converting the IR into MIPS code, but you do not need to use this code in this exercise.

The makefile must be located in the following path:
\begin{itemize}
    \item ex4-dfa/Makefile
\end{itemize}
This makefile will build the analyzer (a runnable jar file) in the following path:
\begin{itemize}
    \item ex4-dfa/ANALYZER
\end{itemize}
Feel free to reuse the makefile supplied in the skeleton,
or write a new one if you want to.

\subsection{Command-line usage}
\textit{ANALYZER} receives two parameters (file paths):
\begin{itemize}
    \item \textit{input} (input file path)
    \item \textit{output} (output file path containing the MIPS code)
\end{itemize}

\subsection{Compiling}
To build the skeleton, run the following command (in the \textit{src/ex4-dfa} directory): \\
\texttt{\$ make compile} \\
This performs the following steps:
\begin{itemize}
    \item Generates the relevant files using jflex/cup
    \item Compiles the modules into \textit{ANALYZER}
\end{itemize}

%The skeleton code for this exercise resides (as usual)
%in subdirectory EX4 of the course repository
%(\url{https://github.com/davidtr1037/COMPILATION_TAU_FOR_STUDENTS/tree/master/FOLDER_3_SOURCE_CODE/EX4}).
%COMPILATION/EX4 should contain a makefile building your source files to a
%runnable jar file called COMPILER (note the lack of the .jar suffix).
%Feel free to use the makefile supplied in the course repository,
%or write a new one if you want to. 
%Before you submit, make sure that your exercise compiles and runs
%on the school server: $nova.cs.tau.ac.il$.
%This is the formal running environment of the course.
%
%\paragraph{Execution parameters}
%compiler receives $2$ input file names:\\ \\
%InputProgram.txt\\
%OutputMIPS.s

%\section{Additional Notes}
%\subsection{Global Variables}
%Note that when a global variable is initialized,
%then the initial value is either a constant or another global variable.

\bibliographystyle{unsrt}
\bibliography{ex4-dfa}

\end{document}
